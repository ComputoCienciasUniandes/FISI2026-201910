\documentclass{article}
\usepackage[utf8]{inputenc}
\usepackage[spanish]{babel}
\usepackage{hyperref}
 
\hypersetup{
    colorlinks=true,
    linkcolor=blue,
    filecolor=magenta,      
    urlcolor=blue,
}

\usepackage{listings}
\usepackage{color}

\definecolor{codegreen}{rgb}{0,0.6,0}
\definecolor{codegray}{rgb}{0.5,0.5,0.5}
\definecolor{codepurple}{rgb}{0.58,0,0.82}
\definecolor{backcolour}{rgb}{0.95,0.95,0.92}
\lstdefinestyle{mystyle}{
    backgroundcolor=\color{backcolour}, commentstyle=\color{codegreen}, keywordstyle=\color{magenta},
    numberstyle=\tiny\color{codegray}, stringstyle=\color{codepurple}, basicstyle=\footnotesize,
    breakatwhitespace=false, breaklines=true, captionpos=b, keepspaces=true, numbers=left,                    
    numbersep=5pt, showspaces=false, showstringspaces=false, showtabs=false,tabsize=2
}
\lstset{style=mystyle}

\title{Tarea 11}
\author{fl.gomez10 at uniandes.edu.co}
%\date{March 2019}

\begin{document}

\maketitle

Horario de atención: Principalmente de 2:00pm a 5:00pm en la oficina i-109.
También se pueden enviar dudas al correo electrónico.
Entregar antes de finalizar la clase. 

Trabaje iniciando  sesión en la máquina virtual en línea
\href{https://mybinder.org/v2/gh/ComputoCienciasUniandes/FISI2026-201910/master?urlpath=lab}{mybinder.org/}
\footnote{\url{https://mybinder.org/v2/gh/ComputoCienciasUniandes/FISI2026-201910/master?urlpath=lab}}. 


\section{Ejercicio 1 (20 puntos) Trabajo en Casa - Generar y graficar una distribución de datos Gaussiana (o Normal)}

Vamos a estudiar cómo el tamaño de una muestra afecta lo que vemos de una población.

En un notebook llamado \texttt{hw11.ipynb} genere un conjunto de datos
llamado ``data'' de n=10 datos distribuidos aleatoriamente según
la distribución normal (\texttt{np.random.norm(loc=3.0, scale=1.5, size=n)}),
centrados en 3.0, con un ancho de 1.5.

\begin{itemize}
\item Generar un histograma de los datos con 10, otro con 20
  y otro con 30 columnas usando la opción (bins=10) de la
  función \texttt{plt.hist()}.
\item calcular la media de ``data'' usando \texttt{np.mean(data)}
\item calcular la mediana de ``data'' usando \texttt{np.median(data)}
\item En una celda responda lo siguiente ¿Son cercanas la media y la mediana?
\item ¿Esperaría que fueran iguales la media y la mediana? ¿Por qué?
\end{itemize}

\section{Ejercicio 2 (20 puntos) Trabajo en Casa - Más gaussianas.}

Repita para n = {100, 1000, 10000, 100000}

\section{Ejercicio 3 (40 puntos) En Clase}



\end{document}
