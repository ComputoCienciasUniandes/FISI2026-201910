\documentclass{article}
\usepackage[utf8]{inputenc}
\usepackage[spanish]{babel}
\usepackage{hyperref}
 
\hypersetup{
    colorlinks=true,
    linkcolor=blue,
    filecolor=magenta,      
    urlcolor=blue,
}

\usepackage{listings}
\usepackage{color}

\definecolor{codegreen}{rgb}{0,0.6,0}
\definecolor{codegray}{rgb}{0.5,0.5,0.5}
\definecolor{codepurple}{rgb}{0.58,0,0.82}
\definecolor{backcolour}{rgb}{0.95,0.95,0.92}
\lstdefinestyle{mystyle}{
    backgroundcolor=\color{backcolour}, commentstyle=\color{codegreen}, keywordstyle=\color{magenta},
    numberstyle=\tiny\color{codegray}, stringstyle=\color{codepurple}, basicstyle=\footnotesize,
    breakatwhitespace=false, breaklines=true, captionpos=b, keepspaces=true, numbers=left,                    
    numbersep=5pt, showspaces=false, showstringspaces=false, showtabs=false,tabsize=2
}
\lstset{style=mystyle}

\title{Tarea 12}
\author{fl.gomez10 at uniandes.edu.co}
%\date{March 2019}

\begin{document}

\maketitle

Horario de atención: Principalmente de 2:00pm a 5:00pm en la oficina i-109.
También se pueden enviar dudas al correo electrónico.
Entregar antes de finalizar la clase. 

Trabaje iniciando  sesión en la máquina virtual en línea
\href{https://mybinder.org/v2/gh/ComputoCienciasUniandes/FISI2026-201910/master?urlpath=lab}{mybinder.org/}
\footnote{\url{https://mybinder.org/v2/gh/ComputoCienciasUniandes/FISI2026-201910/master?urlpath=lab}}. 


\section{Ejercicio 1 (40 puntos) Trabajo en Casa - Integrar una función bidimensional con método Monte Carlo.}

Se tiene la función:
\begin{equation}
  f(x,y) = \sin \left(2 \pi x \right) e^{-(x^2 + y^2)/(\pi^2)} + 5
\end{equation}

\begin{itemize}
\item (10 pts) Use \texttt{scipy.integrate.dblquad} para integrar. Este será nuestro valor de referencia.
\item (30 pts) Evaluar la integral en $x \in(-5,5)$ e $y \in (-3,3)$ usando un método Monte Carlo de integración.
  Ajuste su método para tener un error absoluto inferior al $2\%$ respecto al valor de referencia.
\end{itemize}

\section{Ejercicio 2 (60 puntos) Definido En Clase}


\end{document}
