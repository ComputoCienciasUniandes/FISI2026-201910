\documentclass{article}
\usepackage[utf8]{inputenc}
\usepackage[spanish]{babel}
\usepackage{hyperref}
 
\hypersetup{
    colorlinks=true,
    linkcolor=blue,
    filecolor=magenta,      
    urlcolor=blue,
}

\usepackage{listings}
\usepackage{color}

\definecolor{codegreen}{rgb}{0,0.6,0}
\definecolor{codegray}{rgb}{0.5,0.5,0.5}
\definecolor{codepurple}{rgb}{0.58,0,0.82}
\definecolor{backcolour}{rgb}{0.95,0.95,0.92}
\lstdefinestyle{mystyle}{
    backgroundcolor=\color{backcolour}, commentstyle=\color{codegreen}, keywordstyle=\color{magenta},
    numberstyle=\tiny\color{codegray}, stringstyle=\color{codepurple}, basicstyle=\footnotesize,
    breakatwhitespace=false, breaklines=true, captionpos=b, keepspaces=true, numbers=left,                    
    numbersep=5pt, showspaces=false, showstringspaces=false, showtabs=false,tabsize=2
}
\lstset{style=mystyle}

\title{Tarea 09}
\author{fl.gomez10 at uniandes.edu.co}
%\date{March 2019}

\begin{document}

\maketitle

Horario de atención: Principalmente de 2:00pm a 5:00pm en la oficina i-109.
También se pueden enviar dudas al correo electrónico.
Entregar la carpeta de trabajo en un archivo comprimido \texttt{hw09-username.tar}
antes de finalizar la clase. 

Trabaje iniciando  sesión en la máquina virtual en línea
\href{https://mybinder.org/v2/gh/ComputoCienciasUniandes/FISI2026-201910/master?urlpath=lab}{mybinder.org/}
\footnote{\url{https://mybinder.org/v2/gh/ComputoCienciasUniandes/FISI2026-201910/master?urlpath=lab}}. 


\section{Ejercicio 1 (60 puntos) Trabajo en Casa - Ceros de los Polinomios de Legendre}

Los polinomios de Legendre
\footnote{Pueden echarle una ojeada a Wikipedia en \url{https://en.wikipedia.org/wiki/Legendre_polynomials}}
son muy importantes en física para describir problemas descritos en coordenadas
esféricas. Se utilizan en la solución de la función de onda del átomo de Hidrógeno
en Mecánica Cuántica.
En este ejercicio utilizaremos una librería que calcula los polinomios de Legendre.
Esta tarea está fuertemente basada en el vídeo 12.

\subsection{Importando la librería SciPy (10 pts)}
Esta librería tiene herramientas de alto turmequé en física.
Vamos a invocar los polinomios de Legengre $P_m(x)$ con $m=6$.

Inicie un nuevo notebook con el botón a la derecha (new $\rightarrow$ Python3).
En la primera celda importe numpy, matplotlib.plt y scipy.special

\begin{lstlisting}[language=Python]
import numpy as np
import matplotlib.pyplot as plt
import scipy.special as sp
\end{lstlisting}

En la segunda celda un array tipo linspace $x$ entre -1 y 1.
Genere un segundo array $y=P_m(x)$, con $m = 6$.
\begin{lstlisting}[language=Python]
x = np.linspace(-1,1)
y = sp.eval_legendre(6,x)
\end{lstlisting}

Grafique en la tercera celda $x$ vs $y$.
\begin{lstlisting}[language=Python]
plt.title(Polinomio de Legendre de orden 6)
plt.plot(x,y)
plt.grid(True)
plt.axhline(y=0, color="black")
plt.axvline(x=0, color="black")
\end{lstlisting}

\subsection{Buscando ceros por barrido (25 pts)}
Implemente un algoritmo para buscar ceros a través de cambio de signo
en la función mientras se hace barrido en $x$ desde -1 hasta 1.

\begin{itemize}
\item El algoritmo debe barrer el dominio de $x$ en pequeños pasos pequeños.
\item El algoritmo debe decir si encuentra ceros a partir del cambio de
  signo de la función
\item El algoritmo debe encontrar los ceros con un error no mayor a 0.01
\end{itemize}
Preferiblemente, que no sea una copia del código en el vídeo. Entienda
el método y escriba su propia versión.

\subsection{Buscando ceros por bisección (25 pts)}
\begin{itemize}
\item El algoritmo buscará en todos los ceros que encontró en el punto anterior.
\item Si se tiene un cero en $x_0$, el algoritmo  puede empezar a buscar
  dentro de un rango $(x_0 - 0.05 , x_0 + 0.05)$
  alrededor de cada cero que había encontrado.
\item El algoritmo debe dividir el rango entre dos partes, e identificar
  en qué momento cambia la función de signo para aceptar el rango
  donde se hace el cambio de signo y rechazar el dominio donde no
  cambió la función de signo. Esta parte debe ser repetirse hasta
  encontrar el cero dentro del error propuesto (0.00001)
  $|f(x)| < 0.00001$
\item El algoritmo debe encontrar los ceros que encontró en el punto
  anterior.
\item El algoritmo debe reportar cuántas iteraciones requiere para
  encontrar cada uno de los ceros en el dominio.
\end{itemize}
Preferiblemente, que no sea una copia del código en el vídeo. Entienda
el método y escriba su propia versión.

\section{Ejercicio 2: Método de Newton-Raphson. (40 pts) Trabajo en Clase}

Escribir un algoritmo que busque los ceros del polinomio de Legendre
de sexto orden $P_6(x)$ entre -1 y 1.

\begin{itemize}
\item El algoritmo debe utilizar los mismos puntos de partida que en el
  punto anterior, cercanos a los ceros.
\item El algoritmo debe calcular la derivada de la función en el punto
\item A partir de la derivada, debe buscar el intersecto de la recta tangente
  al punto donde se calculó la derivada.
\item Una vez se tiene el intersecto, se evalúa la función, se repite el
  procedimiento hasta que el error sea inferior a 0.00001.
  $|f(x)| < 0.00001$
\item Indicar cuántas iteraciones se requieren para que este método
  encuentre cada uno de los ceros.
\end{itemize}

\end{document}
