\documentclass{article}
\usepackage[utf8]{inputenc}
\usepackage[spanish]{babel}
\usepackage{hyperref}
 
\hypersetup{
    colorlinks=true,
    linkcolor=blue,
    filecolor=magenta,      
    urlcolor=blue,
}

\usepackage{listings}
\usepackage{color}

\definecolor{codegreen}{rgb}{0,0.6,0}
\definecolor{codegray}{rgb}{0.5,0.5,0.5}
\definecolor{codepurple}{rgb}{0.58,0,0.82}
\definecolor{backcolour}{rgb}{0.95,0.95,0.92}
\lstdefinestyle{mystyle}{
    backgroundcolor=\color{backcolour}, commentstyle=\color{codegreen}, keywordstyle=\color{magenta},
    numberstyle=\tiny\color{codegray}, stringstyle=\color{codepurple}, basicstyle=\footnotesize,
    breakatwhitespace=false, breaklines=true, captionpos=b, keepspaces=true, numbers=left,                    
    numbersep=5pt, showspaces=false, showstringspaces=false, showtabs=false,tabsize=2
}
\lstset{style=mystyle}

\title{Tarea 10}
\author{fl.gomez10 at uniandes.edu.co}
%\date{March 2019}

\begin{document}

\maketitle

Horario de atención: Principalmente de 2:00pm a 5:00pm en la oficina i-109.
También se pueden enviar dudas al correo electrónico.
Entregar antes de finalizar la clase. 

Trabaje iniciando  sesión en la máquina virtual en línea
\href{https://mybinder.org/v2/gh/ComputoCienciasUniandes/FISI2026-201910/master?urlpath=lab}{mybinder.org/}
\footnote{\url{https://mybinder.org/v2/gh/ComputoCienciasUniandes/FISI2026-201910/master?urlpath=lab}}. 


\section{Ejercicio 1 (50 puntos) Trabajo en Casa - Ajuste función polinómica}

\begin{lstlisting}[language=Python]
import numpy as np
import matplotlib.pyplot as plt
from scipy.optimize import curve_fit

### Cargar y visualizar datos

data = np.loadtxt("dos_picos_2.dat")
x = data[:,0]
y = data[:,1]
plt.scatter(x,y)


### Ajuste de curvas

def f(x, a, b):
    y = a + b * x
    return y

curve_fit(f,x,y)
\end{lstlisting}

El fragmento anterior de código hace un ajuste de una recta al conjunto de datos
\texttt{dos\_picos\_2.dat}\footnote{\url{https://github.com/ComputoCienciasUniandes/FISI2026-201910/raw/master/Talleres/Grupo_2/dos_picos_2.dat}}, retornando el valor de los parámetros $a$ y $b$.

Modifique el código para que realice el ajuste a un polinomio de
grado no mayor que 10.(20 pts).

Muestre los parámetros óptimos (10pts).

Grafique la curva de mejor ajuste junto con los datos originales (20pts).

\section{Ejercicio 2: (50 pts) Trabajo en Clase}

Se define en clase.

\end{document}
