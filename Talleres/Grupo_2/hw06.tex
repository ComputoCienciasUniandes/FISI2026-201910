\documentclass{article}
\usepackage[utf8]{inputenc}
\usepackage[spanish]{babel}
\usepackage{hyperref}
 
\hypersetup{
    colorlinks=true,
    linkcolor=blue,
    filecolor=magenta,      
    urlcolor=blue,
}

\usepackage{listings}
\usepackage{color}

\definecolor{codegreen}{rgb}{0,0.6,0}
\definecolor{codegray}{rgb}{0.5,0.5,0.5}
\definecolor{codepurple}{rgb}{0.58,0,0.82}
\definecolor{backcolour}{rgb}{0.95,0.95,0.92}
\lstdefinestyle{mystyle}{
    backgroundcolor=\color{backcolour}, commentstyle=\color{codegreen}, keywordstyle=\color{magenta},
    numberstyle=\tiny\color{codegray}, stringstyle=\color{codepurple}, basicstyle=\footnotesize,
    breakatwhitespace=false, breaklines=true, captionpos=b, keepspaces=true, numbers=left,                    
    numbersep=5pt, showspaces=false, showstringspaces=false, showtabs=false,tabsize=2
}
\lstset{style=mystyle}

\title{Tarea 06}
\author{fl.gomez10 at uniandes.edu.co}
%\date{March 2019}

\begin{document}

\maketitle

Horario de atención: Principalmente de 2:00pm a 5:00pm en la oficina i-109. También se pueden enviar dudas al correo electrónico.

Entregar la carpeta de trabajo en un archivo comprimido \texttt{hw06-username.tar} antes de finalizar la clase. 

Trabaje iniciando  sesión en la máquina virtual en línea
\href{https://mybinder.org/v2/gh/ComputoCienciasUniandes/FISI2026-201910/master?urlpath=lab}{mybinder.org/}
\footnote{\url{https://mybinder.org/v2/gh/ComputoCienciasUniandes/FISI2026-201910/master?urlpath=lab}}. 
Aparte, cree un archivo de texto llamado \texttt{bitacora.txt}


\section*{Ejercicio 1 (30 puntos) Trabajo en Casa}
\subsection*{a (10 pts): Módulo Desconocido.}

\begin{itemize}
    \item Abra un Notebook de Python 3. Guárdelo como \texttt{ejercicio01a.ipynb}
    \item En la primera celda importe el paquete \texttt{imageio}.
    \item Ejecute la celda con \texttt{shift + enter}
    \item ¿Qué resultado arroja la ejecución de esta celda? Guarde la salida (Esta puede ser una advertencia o un error) en la bitácora.
    \item Guarde este notebook. \texttt{ctrl + s}
\end{itemize}

\subsection*{b (20 pts): Instalación de un Módulo con PIP}
\begin{itemize}
    \item Cierre el notebook anterior. Abra una terminal e instale con el comando \texttt{pip} el paquete \texttt{imageio}, del mismo modo que en el \href{https://www.youtube.com/watch?v=IpyG-1Ied3w&list=PLHQtzvthdVM_MGC9dPFKe4hPAwBd_7RJ3&index=10&t=248s}{vídeo (minuto 4:08)} se instala \texttt{jupyter}. 
    \item Guarde en la bitácora la información que arroja el proceso de instalación.
    \item Abra un Notebook de Python 3.
    \item En la primera celda importe el paquete \texttt{imageio}.
    \item Ejecute la celda con \texttt{shift + enter}
    \item ¿Qué resultado arroja la ejecución de esta celda? Guarde la salida (Esta puede ser una advertencia o un error) en la bitácora. Comente este resultado.
    \item Guarde este notebook como \texttt{ejercicio01b.ipynb}
\end{itemize}

\section*{Ejercicio 2 (40 puntos) Trabajo en Casa}

\subsection*{a (10 pts)}
Guarde el siguiente código en un archivo llamado \texttt{grafica\_cos.py}.

\begin{lstlisting}[language=Python, caption=grafica-cos.py]
import matplotlib.pyplot as plt
import math

t = 0
n = 0

A = list( range(0,100))
x = []
for elemento in A:
    x.append( t + elemento * 2 * 3.1416 / 100 )

y = []
for elemento in x:
    y.append( math.cos(elemento))
    
plt.subplot(111)
plt.axis("equal")
plt.plot(x,y)    
plt.title('t = {}'.format(t))
plt.savefig('{}.png'.format(n))
plt.close()
\end{lstlisting}

Este código genera cien números igualmente espaciados entre 0 y $2 \pi$, le suma un número ``t'', luego calcula $y = \cos(x+t)$, grafica y guarda como \texttt{n.png}.

\begin{itemize}
    \item Corra el código una vez para comprobar su funcionamiento.
    \item Cambie $t$ y $n$ (entero) para comprobar su funcionamiento.
    \item Modifique el código para que sea una función que tenga dos argumentos de entrada: $t$ y $n$. Con esto, al llamar la \texttt{funcion(t,n)} esta genera la gráfica y la guarda.
    \item Guarde este archivo.
\end{itemize}

\subsection*{b (20 pts)}
Cree un nuevo Notebook llamado \texttt{ejercicio02b.ipynb}.

\begin{itemize}
    \item Importe el módulo \texttt{grafica\_cos} que creó en el numeral anterior. \footnote{Pilas con el nombre del módulo. Si lo llaman ``grafica-cos'' el programa entenderá que hay que realizar una resta ``grafica'' menos ``cos''. }
    \item Llame la función que creó en el numeral anterior para comprobar que funciona.
    \item Inicie un contador $n=0$, la variable $t=0$ y un paso de tiempo 
    \texttt{DeltaT=0.2}
    \item Cree un ciclo while que termine cuando $t >6.29$
    \item en este ciclo llame la función que grafica con los argumentos t y n.
    \item aumente t += deltat.
    \item aumente n += 1
\end{itemize}
 
Ejecute la celda. Esto debería generar cerca de 30 archivos de imágen. Guarde el Notebook.

\subsection*{c (10 pts)}

En este ejercicio se va a hacer una ``película.gif'' con las imágenes que se crearon en el numeral anterior.

\begin{itemize}
    \item Cree un nuevo notebook llamado \texttt{ejercicio02c.ipynb}.
    \item Cree una lista con los nombres de las imágenes como strings. Llámela  \texttt{filenames}.
    \item Use una estructura de código similar a la propuesta para que el módulo \texttt{imageio} genere un nuevo archivo apilando las imágenes.
\end{itemize}




\begin{lstlisting}[language=Python, caption=ejercicio02x.ipynb]
import imageio

filenames = []  ## Lista con los nombres de los archivos

with imageio.get_writer('pelicula.gif', mode='I') as writer:
    for filename in filenames:
        image = imageio.imread(filename)
        writer.append_data(image)
\end{lstlisting}


\section*{Ejercicio 3 (En Clase)}

Descargue el Notebook \href{https://github.com/ComputoCienciasUniandes/FISI2026-201910/raw/master/Talleres/Grupo_2/colisiones.ipynb}{colisiones.ipynb}

\subsection*{a (30 pts)}

\begin{itemize}
    \item Añada las posiciones $X$ e $Y$ de la partícula $c$ a las listas. Grafique la posición de $c$ junto con las demás partículas. (10 pts)
    \item Añada interacción entre la partícula $c$ y las ya creadas $a$ y $b$. (10 pts)
    \item Coloque $c$ en algún punto donde $x_c < -2$ y modifique la posición $y$ inicial de $a$ para que impacte $a$ primero con $b$ y luego con $c$. (10 pts)
    
    
\end{itemize}

\subsection*{b (30 pts) BONUS}
\begin{itemize}
    \item Muestre si se conserva el momento lineal del sistema graficando las componentes $x$ e $y$ del momento total (20 pts.)
    \item Grafique la energía cinética total del sistema. (10 pts.)
    ¿Se comporta como esperaba?
\end{itemize}

\end{document}
