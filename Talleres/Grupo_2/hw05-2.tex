\documentclass{article}
\usepackage[utf8]{inputenc}
\usepackage[spanish]{babel}
\usepackage{amsmath}

 
\usepackage{listings}
\usepackage{color}
 
\definecolor{codegreen}{rgb}{0,0.6,0}
\definecolor{codegray}{rgb}{0.5,0.5,0.5}
\definecolor{codepurple}{rgb}{0.58,0,0.82}
\definecolor{backcolour}{rgb}{0.95,0.95,0.92}
\lstdefinestyle{mystyle}{
    backgroundcolor=\color{backcolour},   
    commentstyle=\color{codegreen},
    keywordstyle=\color{magenta},
    numberstyle=\tiny\color{codegray},
    stringstyle=\color{codepurple},
    basicstyle=\footnotesize,
    breakatwhitespace=false,         
    breaklines=true,                 
    captionpos=b,                    
    keepspaces=true,                 
    numbers=left,                    
    numbersep=5pt,                  
    showspaces=false,                
    showstringspaces=false,
    showtabs=false,                  
    tabsize=2
}
\lstset{style=mystyle}

\title{Tarea 05}
\author{fl.gomez10 at uniandes.edu.co}
%\date{March 2019}

\begin{document}

\maketitle

\section*{Entrega}
En un archivo comprimido .tar antes de finalizar la clase. 

\section*{Ejercicio 1 (50 puntos)}

En el archivo \texttt{ejercicio01.py} vamos a trabajar el problema de objetos 
celestes recorriendo el sistema solar. Este problema está basado en el vídeo
de Programación Oientada a Objetos en Python.


Vamos calcular la órbita de la Tierra 
a partir de la posición y la velocidad inicial. Para simplificar el problema 
vamos a trabajar en dos dimensiones. No todos los cuerpos en el sistema solar 
se mueven en el mismo plano, pero para esta tarea no será relevante estudiar 
esto en detalle.

Generalmente en aplicaciones de física se trabaja con las unidades del sistema
internacional (metro, kilogramo, segundo), sin embargo en este problema tenemos
masas de varias veces millones de toneladas, distancias de miles de millones de 
kilómetros y tiempos muy largos (meses y años). Así que adoptaremos las 
siguientes unidades:


\begin{itemize}
    \item Unidad astronómica de distancia (Distancia Tierra-Sol):  
    $ 1 \mathrm{a.u.} \sim 1.45 \times 10 ^8\mathrm{km}$
    \item Unidad de tiempo: $1 \mathrm{yr} \sim 3.15 \times 10^7 \mathrm{s}$
    \item Unidad de Masa (Masa Solar): $1 M_{\odot} \sim 333000$ Masas terrestres
     $\sim 2 \times 10^{30} \mathrm{kg}$
\end{itemize}

Con esto, la velocidad que tiene la Tierra en su traslación alrededor del Sol es
de $6.28 \mathrm{u.a./yr}$ (perímetro del "círculo" sobre el periodo de órbita, 
unos $29.8 \mathrm{km/s}$.)

La constante de gravitación $G$ queda:
\[ G = 39.48   \frac{\mathrm{u.a.}^3}{ \mathrm{yr} ^ 2  M_\odot }\]

\subsection*{a. Inicia atributos (5 pts)}
Siguiendo el ejemplo del vídeo, vamos a crear una clase llamada CuerpoCeleste con 
los siguientes atributos:
\begin{itemize}
    \item $x$, $y$: posición en unidades astronómicas
    \item $v_x$, $v_y$: velocidad en unidades astronómicas por año
    \item $M$: masa en masas solares
    \item $F_x$, $F_y$: Fuerza gravitacional que actúa sobre el objeto. En masas
    solares por unidad astronómica sobre año al cuadrado. 
\end{itemize}

\subsection*{b. Calcula Fuerza (30 pts)}

Implementar un método llamado ``calculeFuerza'' que calcule la fuerza de
atracción gravitacional que experimenta el cuerpo celeste debido al Sol. 
Este método va a 
cambiar los valores de $F_x$ y $F_y$.

La fuerza gravitacional que experimenta el cuerpo 1 debido al cuerpo 2 se puede
calcular como:
\[ \vec{F}_{1,2} = - G \frac{M_1 \times{M_2}}{R^2} \hat{r}_{2,1}\]
con $\hat{r}_{2,1}$ el vector unitario que va de 2 hacia 1.

Si elegimos al Sol como $M_2 = 1$ unidad de masa solar y lo colocamos en el origen de las coordenadas $\vec{r}_2 = 0$, queda:
\[ \vec{F}_{1} = - G \frac{M_1}{R^2} \hat{r}_{1}\]

Finalmente, la fuerza que experimenta el cuerpo celeste debido al Sol se puede esribir como:
\[ \vec{F}_1 = \left( F_x, F_y\right) \]

\subsection*{c. Muevete(10 pts)}
Implementar un método llamado "muevete(dt)" que dependa de un $\delta t$. Este método
desplaza la partícula y calcula el cambio de velocidad según la fuerza experimentada.

\subsection*{d. Evoluciona e imprime (5 pts)}

\begin{itemize}
    \item Cree un cuerpo celeste llamado Tierra con $x=1$, $y=0$, $v_x=0$, $v_y=6.28$ y 
$M = 1 / 333000$
    \item Utilice un intervalo de tiempo $\delta t  = 1 / 365$
    \item Haga el sistema evolucionar desde $t=0$ hasta $t=.25$ (esto es la cuarta 
    parte de un año) imprimiendo t, x e y.
    \item ¿Cuál es la posición final?
\end{itemize}

\section*{Ejercicio 2 (20 puntos)}

Una vez tenga el ejercicio anterior completo, copielo en un nuevo archivo llamado
``jercicio02.py''. 

Cree un par de listas de posición $X=$[ ] y posición $Y=$[ ], en estas 
va a guardar la posición cada vez que ejecute "tierra.muevete(dt)"

Después que el sistema haya terminado de evolucionar, incluya estas líneas en su 
código para generar una gráfica de la trayectoria.

\begin{lstlisting}[language=Python, caption=Python example]
import matplotlib.pyplot as plt
fig = plt.figure(figsize=(5,5))
plt.plot(X,Y)
plt.savefig('orbita.png')
\end{lstlisting}

Con esto se genera un archivo de imágen con el segmento de órbita calculado.

\subsection*{a (10pt)}
Grafique la trayectoria desde t=0 hasta t = 0.25. Modifique el nombre del arguivo de salida para que no sea reescrito en el siguiente item.
\subsection*{b (10 pt)}
Grafique la trayectoria desde t = 0 hasta t = 0.75, Cambie el nombre del archivo de salida.

\section*{Ejercicio 3 (30 pts)}
Una pelota caucho tiene una masa de $15\mathrm{g}$ y
un radio de $2.6\mathrm{cm}$. Asuma $g = 980 \mathrm{cm/s^2}$.

Esta se deja caer sobre una superficie dura ubicada en $y=0$
Puede modelar la fuerza de contacto como:

\[ F_y =  - k |R - y |^3\]

con $k = 0.0005$ dinas*$(\mathrm{m})^{-3})$, R el radio de la pelota
e $y$ la posiciòn del centro de masa de la pelota, de modo que 
cuando la pelota entra en contacto con la superficie $y=R$.


El código va a correr dos experimentos.

\subsection*{Experimento 1 (15pts)}

\begin{table}[ht]
    \centering
    \begin{tabular}{c|c}
         DeltaT & 0.001 s \\
         Tiempo Inicial & 0 s \\
         Altura Inicial &  50 cm\\
         Velocidad Inicial & 0 cm/s $\hat{i}$ + 0 cm/s $\hat{j}$ 
    \end{tabular}
    \caption{Condiciones iniciales Experimento 1.}
    \label{tab:experimento1}
\end{table}

(10 pts): Graficar altura (eje vertical) en función del tiempo (eje horizontal) desde t=0 hasta el segundo rebote.
Guardar como ``figura1.png''.


(5 pts): Graficar la trayectoria (Y vs X. Guardar como ``figura1.png''.

Lol.

\subsection*{Experimento 2 (15pts)}

\begin{table}[ht]
    \centering
    \begin{tabular}{c|c}
         DeltaT & 0.001 s \\
         Tiempo Inicial & 0 s \\
         Altura Inicial &  50 cm\\
         Velocidad Inicial & 30 cm/s $\hat{i}$ + 40 cm/s $\hat{j}$ 
    \end{tabular}
    \caption{Condiciones iniciales Experimento 2.}
    \label{tab:experimento2}
\end{table}


(10 pts): Graficar altura (eje vertical) en función del tiempo (eje horizontal) desde t=0 hasta el segundo rebote. Guardar como ``figura3.png''.

(5 pts): Graficar la trayectoria (Y vs X) hasta el segundo rebote.
Guardar como ``figura4.png''.

Lol.

\section*{Bonus Track (40 extra pts)}

La magnitud de la fuerza de colisión entre dos pelotas de billar puede modelarse como:
\[  F = \left\{ \begin{array}{ll}
     0, & \text{Si no hay contacto}\\
     h (2R -  |\vec{R_2} - \vec{R_1}|^3), & \text{Cuando hay contacto.}
\end{array} 
\]con $h \sim 10^5 \mathrm{dina/cm^3}$. Cree dos pelotas de radio 2.5 cm y masa 150 g. Grafique la trayectoria de las dos desde $t=0$ hasta $t=10$

Condiciones iniciales:

\begin{table}[hb]
    \centering
    \begin{tabular}{l|r|c}
        Atributo & A & B \\ \hline
        $x_0$ (cm) & $-5.0$ & $0.0$ \\
        $y_0$ (cm) & $1.0$ & $0.0$ \\
        $v_{x0}$ (cm/s) & $+20.0$ & $0.0$ \\
        $v_{y0}$ (cm/s) & $0.0$ & $0.0$ \\
     \end{tabular}
    \caption{Condiciones iniciales de las bolas A y B para el bono.}
    \label{tab:my_label}
\end{table}




\section*{Anexo: Estructura del código.}
Esta puede ser una guía de la estructura del código a implementar para los ejercicios 01 y 02.


\begin{lstlisting}[language=Python, caption=Estructura del ejercicio 2.]

#!/usr/bin/python

class CuerpoCeleste:
    
    def __init__(self, x0,y0,Vx0, Vy0, M0):
        # Parte a. Inicia atributos
        
    def calculaFuerza(self):
        # Parte b. Calcula Fuerza (30 pts)
        G = 39.5
        M_sun = 1 # Solar Mass
        
        R = # Calcular R
        
        self.Fx =  # Calcular Fuerza en x
        self.Fy =  # Calcular Fuerza en y
            
    def muevete(self, dt):
        # Parte c. Muevete (10 pts)
        
Tierra = CuerpoCeleste( 1.0, 0.0, 6.28 , 0.0 , 1/333000)
Tierra.calculaFuerza()

t=0.0
DeltaT = 1 / 365 # one day

X = []
Y = []

while t < 0.25 :
    Tierra.calculaFuerza()
    Tierra.muevete(DeltaT)
    t += DeltaT
    # Ejercicio 2 (X.append(algo), Y.append(algo))
    
import matplotlib.pyplot as plt

fig = plt.figure(figsize=(5,5))
plt.plot(X,Y)
plt.title("Titulo de la grafica")
plt.xlabel("Etiqueta eje horizontal")
plt.ylabel("etiqueta eje vertical")
plt.savefig('figura1.png')
\end{lstlisting}

\end{document}

opllllllllllllllllllll||ttttttttt+ḱ.vg...-----------------------klllt´´´´´FFFFFFFFFFFFFFFFFFFFFFFFFFFFFFFFFFFFFFFFFFFFFFFFFFFFFFFFFFFFFFFFFFFFFFFFFFFFFFFFFFFFFFFFFFFFFFFFFFFFFFFFFFFFFFFFFFFFFFFFFFFFFKk-L
