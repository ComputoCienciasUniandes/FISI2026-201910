\documentclass{article}
\usepackage[utf8]{inputenc}
\usepackage[spanish]{babel}
\usepackage{hyperref}
 
\hypersetup{
    colorlinks=true,
    linkcolor=blue,
    filecolor=magenta,      
    urlcolor=blue,
}

\usepackage{listings}
\usepackage{color}

\definecolor{codegreen}{rgb}{0,0.6,0}
\definecolor{codegray}{rgb}{0.5,0.5,0.5}
\definecolor{codepurple}{rgb}{0.58,0,0.82}
\definecolor{backcolour}{rgb}{0.95,0.95,0.92}
\lstdefinestyle{mystyle}{
    backgroundcolor=\color{backcolour}, commentstyle=\color{codegreen}, keywordstyle=\color{magenta},
    numberstyle=\tiny\color{codegray}, stringstyle=\color{codepurple}, basicstyle=\footnotesize,
    breakatwhitespace=false, breaklines=true, captionpos=b, keepspaces=true, numbers=left,                    
    numbersep=5pt, showspaces=false, showstringspaces=false, showtabs=false,tabsize=2
}
\lstset{style=mystyle}

\title{Tarea 08}
\author{fl.gomez10 at uniandes.edu.co}
%\date{March 2019}

\begin{document}

\maketitle

Horario de atención: Principalmente de 2:00pm a 5:00pm en la oficina i-109.
También se pueden enviar dudas al correo electrónico.
Entregar la carpeta de trabajo en un archivo comprimido \texttt{hw08-username.tar}
antes de finalizar la clase. 

Trabaje iniciando  sesión en la máquina virtual en línea
\href{https://mybinder.org/v2/gh/ComputoCienciasUniandes/FISI2026-201910/master?urlpath=lab}{mybinder.org/}
\footnote{\url{https://mybinder.org/v2/gh/ComputoCienciasUniandes/FISI2026-201910/master?urlpath=lab}}. 


\section{Ejercicio 1 (30 puntos) Trabajo en Casa - Plot \& Scatter}

Trabaje en el notebook ``ejercicio01.ipynb''.

\begin{itemize}
\item Genere un linspace ``t''.

\item Grafique \texttt{plt.plot(t, np.cos(t))}
  
\item (24 pts) En una sola celda se van a generar seis gráficos con el siguiente snippet:
\begin{lstlisting}[language=Python]
f = plt.figure(figsize=(10,10))
ax1 = f.add_subplot(321)
ax2 = f.add_subplot(322)
ax3 = f.add_subplot(323)
ax4 = f.add_subplot(324)
ax5 = f.add_subplot(325)
ax6 = f.add_subplot(326)

ax1.plot(t, np.cos(t))
ax2.plot(t, np.sin(t))
ax3.plot(np.cos(t), np.sin(t))
ax4.scatter(t, np.cos(t))
ax5.scatter(t, np.sin(t))
ax6.scatter(np.cos(t), np.sin(t))
\end{lstlisting}
\item (2 pts) En una nueva celda cambie los parámetros de \texttt{figsize} y grafique de nuevo.
\item (2 pts) En una nueva celda cambie todos los $n$ subplots a \texttt{axn = f.add\_subplot(33n)}
  (de 32n a 33n), excepto el último, déjelo como \texttt{ax6 = f.add\_subplot(329)}
\item (2 pts) Cambie todos los $n$ subplots a \texttt{axn = f.add\_subplot(24n)}
(de 32n a 24n), excepto el primero, déjelo como \texttt{ax1 = f.add\_subplot(247)}
\end{itemize}

\section{Ejercicio 2 (30 puntos) Trabajo en Casa - Scatter 3D}

Inicie el notebook ``ejercicio02.ipynb'' con:
\begin{lstlisting}[language=Python]
import numpy as np
import matplotlib.pyplot as plt
from mpl_toolkits.mplot3d import Axes3D
\end{lstlisting}

\begin{itemize}

\item Genere un par de arrays $s, t$  tipo \texttt{linspace} desde cero hasta $2\pi$

\item Con estos, genere un par de arrays $x$ e $y$ usando \texttt{meshgrid} como:
\begin{lstlisting}[language=Python]
x, y = np.meshgrid(s,t)
\end{lstlisting}

\item Escriba una función
  \begin{equation}
    f(x,y) =  e^{-x/\pi}  \cos x \sin y
  \end{equation}
\item Con la función genere un array $z = f(x,y)$

\item (10 pts) Imprima la forma (\texttt{shape}) de los arrays $s, t, x, y, z$.
  Explique que hace la función \texttt{meshgrid}.
  
  
\item (20 pts) Genere una gráfica con:
\begin{lstlisting}[language=Python]
fig = plt.figure()
ax = fig.add_subplot(111, projection="3d")

ax.scatter(x, y, z)

ax.set_xlabel("X Label")
ax.set_ylabel("Y Label")
ax.set_zlabel("Z Label")

plt.show()
\end{lstlisting}
\end{itemize}



\section{Ejercicio 03 - Plot Surface (3D) \& Plot Color Mesh (2D)}

\subsection{A (20 pts)}
Genere una gráfica 3D con \texttt{ax.plot\_surface(x, y, z)} con los mismos arrays $x$ e $y$ del ejercicio 2.
\subsection{B (20 pts)}
Genere una gráfica 2D con \texttt{ax.pcolormesh(x, y, z)} con los mismos arrays $x$ e $y$ del ejercicio 2.
Use \texttt{ax = fig.add\_subplot(111)} sin el argumento \texttt{projection="3d"}.

\end{document}
